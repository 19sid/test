\documentclass[]{IEEEphot}

\jvol{xx}
\jnum{xx}
\jmonth{June}
\pubyear{2009}

\newtheorem{theorem}{Theorem}
\newtheorem{lemma}{Lemma}

\begin{document}

\title{A Proxemic Multimedia Interaction over the
Internet of Things}

\author{Siddharth Banra}

\affil{MCRLab, University of Ottawa, Ottawa ON K1N 6N5, Canada}  

\doiinfo{aahma078,msain2,elsaddik@uottawa.ca \break
http://www.mcrlab.uottawa.ca/}%

\maketitle

\markboth{IEEE Photonics Journal}{Volume Extreme Ultraviolet Holographic Imaging}



\begin{abstract}
With the rapid growth of online devices, a new concept of
Internet of Things (IoT) is emerging in which everyday devices will be
connected to the Internet. As the number of devices in IoT is increasing,
so is the complexity of the interactions between user and devices. There
is a need to design intelligent user interfaces that could assist users in
interactions. The present study proposes a proximity-based user interface for multimedia devices over IoT. The proposed method employs a
cloud-based decision engine to support user to choose and interact with
the most appropriate device, reliving the user from the burden of enumerating available devices manually. The decision engine observes the
multimedia content and device properties, learns user preferences adaptively, and automatically recommends the most appropriate device to
interact. The system evaluation shows that the users agree with the proposed interaction 70% of the times.
\end{abstract}

\begin{IEEEkeywords}
Proxemic interaction; multimedia interaction ; user interface; elicitation study.
\end{IEEEkeywords}

\section{Introduction}

The Internet has been expanding very rapidly with time. Now, more than 2.7
billion people (almost 39% of world’s population) have access to it and use it in
their daily life [1]. The number of devices that are connected to the Internet has
been growing dramatically. Therefore, a new concept of Internet of Things (IoT)
is emerging [2]. More than 11.2 billion devices were connected to the Internet in
2013 and it is predicted that there will be around 50 billion devices online by 2020
[3]. These uniquely identifiable objects and devices move and interact with each
other to accomplish various tasks. In other words, IoT is like a big and dynamic
society of objects and people. But we are far from the ubiquitous computing
vision of Weiser [4] due to two main reasons: the lack of an appropriate taskcentered User Interface (UI) design approach and the lack of reliable support
for distributable user interfaces in ubiquitous environments [5]. Thus, there is a
need for a new generation of specifically designed user interfaces for IoT.



More than 7 billion people constantly use verbal and nonverbal communication techniques to interact with their society all across the world. Verbal communication is considered as the main channel since it is used explicitly. However, nonverbal communication such as Proxemics, Haptics, Body Language, etc. plays
an important role in the human society because it uses implicit interactions. In
the same way, nonverbal communication can also enhance the interactions within
the IoT, which is ike human’s society except the size of IoT is few times bigger
than human society. Thus, nonverbal techniques should be used in order to build
proper user interfaces for ubiquitous environments, particularly the IoT.\\



The present study proposes a distributed user interface that provides a suitable environment for multimedia interactions over the IoT. It consists of a cloudbased decision engine that manages the proxemic interactions. This engine is
called Proxemic Interaction Unit (PIU). PIU uses multimedia devices within
the IoT as elements of a universal UI in order to assist users to make use of
their surrounding devices. We collected some of the effectual multimedia interaction variables such as device properties, multimedia content attributes, user’s
preferences, etc. to propose a scoring mechanism for the PIU. A group of these
variables are obtained by conducting a user survey and elicitation study while
the rest are extracted from the literature. To further personalize user preferences,
we update the variables after every interaction. The evaluation shows that 70%
of the times the user interface chooses the same device a user would. We reckon
that the user preference learning mechanism will further increase the accuracy
of the user interface.


The rest of this paper firstly presents a review of related works in Section
2. Next, Section 3 describes the architecture and details of proposed UI. The
results of an evaluation user study are given in Section 4, which is followed by
brief discussion in Section 5. This study ends with the conclusion and future
work in Section 6.


\section{ Related Work}

Researchers have been trying to develop new UIs over the IoT for the last few
years. However, the idea of using proxemic interactions has been around for
a while. Hall highlighted the influence of proxemic behavior on interpersonal
communication in 1966 [6]. He divided proxemic interactions into two levels:
micro-level, which studies the way people interact with each other in daily life
and macro-level, which reviews the space organization of houses, buildings and
ultimately towns. More recently, Greenberg et al. proposed a practical version
of proxemic interaction that considers people, digital devices, and non-digital
objects [7]. They defined five dimensions for proxemic interaction: distance, orientation, movement, identity, and location. Every measured change in any dimension can trigger an interaction. Marquardt et al. used this terminology to
develop a proximity toolkit in order to aid fast prototyping of proxemic interactions [8]. Despite the fact that researchers have been trying to define a more
precise structure and terminology for proxemic interactions recently, proxemics
have been involved in applications for a long time.



Researchers study the interactions between smart objects in the IoT (e.g. [9]).
However, the research community has paid more attention to the human-device interactions (e.g. [10, 11]). Ju et al. proposed the Range as a public interactive
whiteboard, which supports co-located, ad-hoc meetings [12]. It uses the proximity sensing to proactively manage the transitions between display and authors
(e.g. to clear the space for writing). Wang et al. introduced Proxemic Peddler,
which is a public display [13]. This public display can capture and preserve the
attention of pedestrians. Both of the aforementioned studies are examples of
human-device proxemic interactions in micro-level where the proximity space
is very small (e.g. one room). Researchers also have developed some proximity
aware applications at macro-level (e.g. multiple rooms). The active badge, which
was introduced by Want et al., is one of them [14]. They embedded an infrared
beacon into their badge in order to connect it to a sensory network. Using this
architecture, they could track employees in the office and redirect their phone
calls to their current stations. Active badge was successfully implemented and
used in large scale.



Moreover, there are some studies that focus on possible human-device proxemic interactions in households. The EasyLiving is one of the earliest projects
in this group [15]. It emphasizes on an architecture, which can connect different
devices together in order to enhance the user experience. Ramani et al. proposed
a new location tracking system for media appliances in the wireless home networks, which can be used to build a proxemic media redirection system [16].
Recently, Sørensen et al. introduced and evaluated the AirPlayer, which is a
multi-room music system that uses the proxemic interaction [17].



In conclusion, there is no custom designed proxemic interface for multimedia
interactions. The proposed UI is designed and optimized for proxemic multimedia interaction, which makes it a task-specific UI. Furthermore, we believe that
a completely distributed solution (e.g. [17]) cannot meet the growing requirements of the IoT users. Therefore, the proposed method employs a cloud-based
decision engine (PIU), which considers additional information such as multimedia content properties, user’s preferences, devise capabilities etc. in order to
improve multimedia interactions. Moreover, the current trend is that a cloudbase database keeps the shared resources including multimedia data. Hence, our
design of a cloud-based decision engine is in accordance with the current technological trends.

\section{ Proposed System}

The main purpose of the proposed UI is to provide an environment within the
IoT that supports the proxemic interactions for multimedia devices. The proposed system has three main components: proxemic interaction unit (PIU), media redirection unit, and tracking engine. PIU is the central part of the system.
Figure 1 presents the abstract algorithm of PIU. As we can see in the figure, the
algorithm begins by receiving the location information of the devices. PIU uses
the tracking engine to find the locations and update it on the location database.
Next, PIU checks for possible proxemic interaction. Then PIU calculates a score
between 0 and 1 for each device based on content properties, user preferences,
and device capabilities. Consequently, the algorithm finds the device with maximum score. If it is not the same device as the one in use, it notifies the user and
recommends a media redirection. If user accepts the recommendation, PIU asks
the media redirection unit to handle this process. After each interaction, PIU
updates the scoring coefficients to learn user’s preferences adaptively.


The device scoring and recommendation step in the aforementioned algorithm is the most important step since it can assist users to engage with a larger
number of devices in the IoT. To recommend the most appropriate device in the
given scenario, we should find and imply different variables in the scoring mechanism. Therefore, we start by introducing these variables and then present the
scoring mechanism. There are three groups involved in the proposed solution:
devices, users, and multimedia content. Below we study these groups in order to
find effective variables for the scoring mechanism.

\subsection{Devices}

Devices are the first and most important objects in our system. Since the proposed system provides an environment for proxemic interactions, we begin by
explaining device’s proximity behavior. Hall defined four perimeters around each
person: intimate space, personal space, social space, and public space (Table 1)
[6]. We are using same categorization for devices according to their effective quality in each space: intimate devices, personal devices, social devices, and public
devices. For example, smartphones have small screens, so they are usually used
by individuals separately. Moreover, they have the best visual quality when they
are close to the user (e.g. 20 cm which is in the intimate space). Users can still
see the smartphone screen when they are farther away, but their effective quality decreases. Similarly, users enjoy big screen TVs most when they are at an
acceptable distance, i.e., 3.5 m to 6 m. Hence, they are placed in the category
of public devices. Let T(x) be a function that returns an integer between 0 to 3
depending on the type of device x. Table 1 provides examples of all 4 groups of
device spaces and function T() value for each device type. We will use function
T() later while calculating scored for each device.

\begin{table}[!t]
\centering
\caption{Hall’s personal space definitions and device examples for each space. \LaTeX}
\label{tab1}
\begin{IEEEeqnarraybox}[\IEEEeqnarraystrutmode\IEEEeqnarraystrutsizeadd{2pt}{1pt}]{v/c/v/r/v/c/v/c/v/}
\IEEEeqnarrayrulerow\\
& \mbox{{\bf Space Name}} && \mbox{{\bf Space Area}} && \mbox{{\bf Example}} && \mbox{{\bf T()}} &\\
\IEEEeqnarraydblrulerow\\
\IEEEeqnarrayseprow[3pt]\\
& \mbox{Intimate Space} && \mbox{distance$\leq$0.45 m} && \mbox{smartphones} && 0 
&\IEEEeqnarraystrutsize{0pt}{0pt}\\
\IEEEeqnarrayseprow[3pt]\\
\IEEEeqnarrayrulerow\\
\IEEEeqnarrayseprow[3pt]\\
& \mbox{Personal Space} && \mbox{0.45 m$\leq$distance $\leq$ 1.2 m} && \mbox{tablets, laptops }&& 1 &\IEEEeqnarraystrutsize{0pt}{0pt}\\
\IEEEeqnarrayseprow[3pt]\\
\IEEEeqnarrayrulerow\\
\IEEEeqnarrayseprow[3pt]\\
& \mbox{Social Space} && \mbox{1.2 m $\leq$distance $\leq$ 3.6 m}  && \mbox{PCs, digital displays}&&2 &\IEEEeqnarraystrutsize{0pt}{0pt}\\
\IEEEeqnarrayseprow[3pt]\\
\IEEEeqnarrayrulerow\\
\IEEEeqnarrayseprow[3pt]\\
& \mbox{Public Space} && \mbox{3.6 m$\leq$distance $\leq$ 7.6 m} && \mbox{TVs, home stereos}&& 3 &\IEEEeqnarraystrutsize{0pt}{0pt}\\
\IEEEeqnarrayseprow[3pt]\\
\IEEEeqnarrayrulerow
\end{IEEEeqnarraybox}
\end{table}


\subsection{User Survey}

It is important to characterize user priorities in order to design effective user
interface. We conducted a user survey to study user’s attitudes and preferences.
Participants had to answer 17 questions regarding the frequency of using multimedia contents, their device preferences, and satisfaction levels. We prepared
questions such that they do not require any specific domain knowledge. Due to
lack of space, we could not present the designed questionnaire here. The website
we used for the study and the questionnaire details can be found at 1
. Totally,
149 people of an average age of 28.62 years participated in our study with the
following gender distribution: 53.7% male and 46.3% female. They were engineers, employees, physicians, university students and professors, etc. with different background levels in IT; 53% of them were involved in a profession that
requires a high level of IT knowledge while the other 47% were not involved in
those kinds of jobs.



Results of this user survey revealed some interesting points. First, we could
not find any correlation between the gender and device preferences. However,
when we grouped our participants into young ($\leq$30 years) and old ($\geq$30 years)
users, we found that old participants are more interested in using TVs and PCs
than tablets and smartphones. In addition to that, profession had an effect on
the device preferences. Participants who were involved in jobs that need a high
level of IT knowledge were keener to use PCs and TVs. We also found that
frequent users (who spend up to 1 hour per day) prefer TVs and PCs more than
non-frequent users non-frequent users (who spend more than 1 hour per day).


To summarize, based on the results of this user survey, we were convinced to
include three characteristics of the engaged user, u, in our scoring mechanism:
participant’s age $(U^{a})$, profession type $(U^{p})$, and multimedia usage habits $(U^{h})$.
We used the user ratings to initialize preference coefficients corresponding to
these characteristics, which are updated over time according to user interactions.



\section{Evaluation}
 To evaluate the proposed system, we defined 4 scenarios and conducted a user
survey. In these scenarios, participants were given access to all types of devices
in their effective distance. The content’s length was the only variable in our
scenarios. In the first scenario, it was 2 min while it was 7 min, 22 min, and 90
min in the second, third and fourth scenarios respectively. As an example, the
user had to select his / her preferred device for watching a 7 min video while he
/ she has access to a smartphone in 0.25 m, tablet in 0.75 m, personal computer
in 2.5 m, and TV in 5 m. Totally, 10 people participated in our study with age
ranging from 20 to 40. The users had different professions and media playback
habits. We compared the user responses to the device recommendations made
by the proposed user interface to measure the usability




We also compare the proposed approach with two other methods. The first
one only uses the distance to suggest a new device. So, the method always suggests the device that has user in its most effective area, with the same learning
mechanism as ours. The second method is AirPlayer [17]. AirPlayer only supports intimate and public devices. Furthermore, it does not have the learning
mechanism. So, it cannot adapt itself to user’s preferences. Table 2 shows the
results of this evaluation study.


In this study, the total accuracy of the first suggestion for our system is
equal to 70% while it was 67.5% for the distance only method and 62.5% for the
AirPlayer. But, since the proposed method can learn from the users responses
and adapt itself to his / her preferences, we decided to study the accuracy of
these methods after two interactions. This is not applicable to the AirPlayer
due to its lack of learning mechanism. The total accuracy after two interactions
for our system was 90%, however it was 85% for the distance only method. To
summarize, we can say that the proposed method had a higher accuracy in these
scenarios. As a result, it could provide an environment that has more successful
proxemic multimedia interactions.

\section*{Acknowledgements}
The authors wish to thank the anonymous reviewers for their valuable suggestions.  

%% \ackrule

\bibliographystyle{IEEEtran}
\bibliography{thesis}

\section*{Biographies}

\textbf{P. W. Wachulak} received the degree${\ldots}$ \\[6pt]
\textbf{M. C. Marconi} received the degree${\ldots}$ \\[6pt]
\textbf{R. A. Bartels} received the degree${\ldots}$ \\[6pt]
\textbf{C. S. Menoni} received the degree${\ldots}$ \\[6pt]
\textbf{J. J. Rocca} received the degree${\ldots}$



\end{document}
# test
